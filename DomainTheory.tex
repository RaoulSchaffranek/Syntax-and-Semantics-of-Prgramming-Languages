\documentclass{article}[12pt]
    \usepackage{amsmath}
    \usepackage{amsfonts}
    \usepackage{tikz}
    \usetikzlibrary{positioning}
    \usepackage{enumitem}
    \usepackage{txfonts}
    \usepackage{float}
    \title{Syntax and Semantics of Programming Languages - Chapter 10: Domain Theory Exercises}
    \author{Raoul Schaffranek}
    \date{September 21, 2018}
    
    \newcommand{\filter}{\ \ensuremath{\text{\textcircled{s}}}\ }
    \newcommand{\lub}{\ensuremath{\textit{}{lub}}}
    \newcommand{\fix}{\ensuremath{\textit{}{fix}}}
    \newcommand{\Fun}{\ensuremath{\text{Fun}}}
    \newcommand{\inS}{\ensuremath{\textit{inS}}}
    \newcommand{\outA}{\ensuremath{\textit{outA}}}
    \newcommand{\head}{\ensuremath{\textit{head}}}
    \newcommand{\tail}{\ensuremath{\textit{tail}}}
    \newcommand{\myif}{\ensuremath{\text{ if }}}
    \newcommand{\then}{\ensuremath{\text{ then }}}
    \newcommand{\myelse}{\ensuremath{\text{ else }}}
    % Hide section numbers
    \renewcommand\thesection{}
    
\begin{document}
\maketitle

\section{Exercise 10.2.6}
Prove that the least upper bound of an ascending chain $x_1 \subseteq x_2 \subseteq x_3 \subseteq \cdots$ of a domain $D$ is unique.
\paragraph{Solution}
Let $y$ and $z$ be two arbitrary but fixed least upper bounds of some ascending chain $x_1 \subseteq x_2 \subseteq x_3 \subseteq \cdots$.
We show that $y = z$.
By the definition of least upper bound, we have that $y \subseteq m$ and $z \subseteq m$ for each upper bound $m$ of $\{x_i\ |\ i \geq 1\}$. In particular, we have that $y \subseteq z$ and $z \subseteq y$. Thus $y = z$.

\section{Exercise 10.2.10}
Prove the following property:\\
A function $f$ in $\Fun(A,B)$ is continuous if and only if both of the following conditions hold.
\begin{enumerate}[label=\alph*)]
    \item $f$ is monotonic
    \item For any ascending chain $x_1 \subseteq x_2 \subseteq x_3 \subseteq \cdots$ in $A$, $f(\lub_A\{x_i|i\geq 1\}) \subseteq \lub_B\{f(x_i)|i\geq 1\}$.
\end{enumerate}

\paragraph{Solution}

Assume $f$ is continuous. This implies, $f$ is monotonic.
By definition of continuous, we further have, that:
$f(\lub_A\{x_i|i\geq 1\}) = \lub_B\{f(x_i)|i\geq 1\}$ for any ascending chain $x_1 \subseteq x_2 \subseteq x_3 \subseteq \cdots$ in $A$. In partialar we have $f(\lub_A\{x_i|i\geq 1\}) \subseteq \lub_B\{f(x_i)|i\geq 1\}$

Now, assume $f$ is monotonic and for any ascending chain $x_1 \subseteq x_2 \subseteq x_3 \subseteq \cdots$ in $A$, $f(\lub_A\{x_i|i\geq 1\}) \subseteq \lub_B\{f(x_i)|i\geq 1\}$.
It remains to show, that for every ascending chain $x_1 \subseteq x_2 \subseteq x_3 \subseteq \cdots$ in $A$ the following property holds: $\lub_B\{f(x_i)|i\geq 1\} \subseteq f(\lub_A\{x_i|i\geq 1\})$.

Sorry.

\section{Exervise 10.2.12}

First $\inS$: We have to consider two cases: $\inS : A \rightarrow A + B$ and $\inS : B \rightarrow A + B$.\\
\begin{itemize}
    \item Case $\inS: A \rightarrow A + B$:\\
    Let $a_1 \subseteq a_2 \subseteq a_3 \subseteq \cdots$ be an ascending chain in $A$.
    We then have:
    \begin{align*}
        &\mathbin{\hphantom{=}} \lub\{\inS(a_i)\ |\ i \geq 1\}\\
        &= \lub\{ \langle a_i,1 \rangle\ |\ i \geq 1 \}\\
        &= \langle \lub\{a_i\ |\ i \geq 1\}, 1 \rangle\\
        &= \inS(\lub\{a_i\ |\ i >= 1\}) 
    \end{align*}
    \item Case $\inS : B \rightarrow A + B$: analogous to previous case.
\end{itemize}

First $\outA$: Let $x_1 \subseteq x_2 \subseteq x_3 \subseteq \cdots$ be an ascending chain in $A + B$.
There are three cases to consider: Either $x_i = \bot$ for all $i \geq 1$. Or there is some index $k$ with $1 \leq k$,
such that for all $i \geq k$ we have $x_i \in A$ or $x_i \in B$. This is because elements of $A$ are not related
to elements of $B$ via $\subseteq$ and vice versa.
\begin{itemize}
    \item Case $x_i = \bot$ for all $i \geq 1$:
    \begin{align*}
        &\mathbin{\hphantom{=}} \lub\{\outA(x_i)\ |\ i \geq 1\}\\
        &= \lub\{\outA(\bot)\ |\ i \geq 1\}\\
        &= \lub\{\bot\ |\ i \geq 1\}\\
        &= \bot\\
        &= \outA(\bot)\\
        &= \outA(\lub\{\bot\ |\ i \geq 1\})\\
        &= \outA(\lub\{x_i\ |\ i \geq 1\})
    \end{align*}
    \item Case $\exists k. 1 \leq k$, s.t. $\forall i. i \geq k. x_i \in B$
    \begin{align*}
        &\mathbin{\hphantom{=}} \lub\{\outA(x_i)\ |\ i \geq 1\}\\
        &=\lub\{\bot\ |\ i \geq 1\}\\
        &=\bot\\
        &=\outA(\lub\{x_i\ |\ i >= k\})\\
        &=\outA(\lub\{x_i\ |\ i >= 1\}
    \end{align*}
    \item Case $\exists k. 1 \leq k$, s.t. $\forall i. i \geq k. x_i \in A$
    \begin{align*}
        &\mathbin{\hphantom{=}} \lub\{\outA(x_i)\ |\ i \geq 1\}\\
        &=\lub\{\outA(x_i)\ |\ i \geq k\}\\
        &=\lub\{x_i\ |\ i \geq k\}\\
        &=\outA(\lub\{x_i\ |\ i \geq k\})\\
        &=\outA(\lub\{x_i\ |\ i \geq 1\})
    \end{align*}
\end{itemize}

\section{Exercise 10.2.13}
Prove that $\head$ and $\tail$ are continuous functions.

\paragraph{Solution}
All the functions used in the definitions of $\head$
and $\tail$ are continuous and function composition
preserves continuity, hence $\head$ and $\tail$ are continuous.

\section{Exercise 10.2.15}
Let $N = \{\bot,0,1,2,3,\cdots\}$ be the elementary domain of natural numbers.
A function $f: N \rightarrow N$ is called \textbf{strict} if $f(\bot) = \bot$.
Consder a function $add1 : N \rightarrow N$ defined by $add(1) = n + 1$ for all $n \in N$ with $n \neq \bot$.
Prove that if $add1$ is monotonic, it must also be strict.

\paragraph{Solution}

Suppose $f$ is monotonic, but not strict.\\
Then $f(\bot) = n$ for some $n \in N$.\\
Then $f(\bot) \subseteq f(n)$, because $\bot \subseteq n$ and $f$ monotonic.\\
But $f(\bot) = n \not\subseteq n+1 = f(n)$, a contraction!

\section{Exercise 10.3.8}
Consider the following functional defined on functions over the natural numbers:
\begin{align*}
    G &: (N \rightarrow N) \rightarrow (N \rightarrow N)\\
    G &= \lambda g. \lambda n.\myif n = 0 \then 2 \myelse g(n)
\end{align*}
\begin{enumerate}[label=\alph*)]
    \item \label{a} Give and justify a recursive definition that corresponds to this functional -
    that is, an operational definition of a function that will be a fixed point of $G$.
    \item Define four different functions, $g_0$, $g_1$, $g_2$, $g_3$, that are fixed
    points of the functional $G$, including the least fixed point, $g_0$. Carefully
    prove that $g_0$ and $g_1$ are fixed points of $G$.
    \item Draw a diagram showing the relationship "is less defined than or equal" between
    these four functions.
    \item Informally describe the oeprational behavior of the recursive definition in part \ref{a}.
    Which of the four fixed-point functions has the closest behavior to the operational view?
\end{enumerate}

\paragraph{Solution a}
Using fixed-point cunstruction, we obtain:
\begin{align*}
    G^0(\bot) &= \bot\\
    G^1(\bot) &= \lambda.\myif n = 0 \then 2 \myelse \bot\\
    G^2(\bot) &= \lambda.\myif n = 0 \then 2 \myelse (\lambda.\myif n = 0 \then 2 \myelse \bot) \bot\\
              &= \lambda.\myif n = 0 \then 2 \myelse (\myif \bot = 0 \then 2 \myelse \bot)\\
              &= \lambda.\myif n = 0 \then 2 \myelse \bot\\
              &= G^1(\bot)
\end{align*}
Hence, $G^1(\bot) = (\lambda.\myif n = 0 \then 2 \myelse \bot)$ is a fixed point of $G$.
The operational definition is $g = \lambda.\myif n = 0 \then 2 \myelse g(n)$
\paragraph{Solution b}
\begin{align*}
g_0 &= \lambda.\myif n = 0 \then 2 \myelse \bot\\
g_1 &= \lambda. 2\\ 
g_2 &= \lambda.\myif n = 0 \then 2 \myelse 3\\
g_3 &= \lambda.\myif n = 0 \then 2 \myelse 4
\end{align*}
We have already seen, that $g_0$ is a fixed point of $G$ in part a).
Let's turn to $g_1$:
\begin{align*}
    G(g_1) &= \lambda.\myif n = 0 \then 2 \myelse (\lambda. 2) \bot\\
           &= \lambda.\myif n = 0 \then 2 \myelse 2\\
           &= \lambda.2 = g_1
\end{align*}

\paragraph{Solution c}
\paragraph{} % This pseudo-paragraph is needed, so that latex prints the following figure below the previous paragraph.
\begin{figure}[H]
    \begin{tikzpicture}
        \node (g0) {$g_0$};
        \node (g2) [above=of g0] {$g_2$};
        \node (g1) [left=of g2] {$g_1$};
        \node (g3) [right=of g2] {$g_3$};
        \draw (g0) -- (g1);
        \draw (g0) -- (g2);
        \draw (g0) -- (g3);
    \end{tikzpicture}
\end{figure}


\paragraph{Solution d}

If $g$ is applied to $0$ it will evaluate to $2$. If $g$ is applied to any other value
it will not terminate. This is closest to the minmal fixed-point $g_0$ of $G$, which
is undefined for values other than $0$. In contrast, $g_1, g_2$ and $g_3$ are more
defined than $g$.

\section{Exercise 10.3.11}
Use fixed-point induction to prove the equality of the following functions in $N \rightarrow N$:
\begin{align*}
    &f(n) = \myif n > 5 \then n - 5 \myelse f(f(n+13))\\
    &g(n) = \myif n > 5 \then n - 5 \myelse g(n+8)
\end{align*}

\paragraph{Solution}
Consider the respective functionals:
\begin{align*}
    &F\ f\ n = \myif n > 5 \then n - 5 \myelse f(f(n+13))\\
    &G\ g\ n = \myif n > 5 \then n - 5 \myelse g(n+8)
\end{align*}
We show $(F\ f\ n = G\ g\ n)$ by fixed-point induction on $f$ and $g$:
\begin{itemize}
    \item Case $\bot, \bot$:
    \begin{align*}
        F\ \bot\ n &= \myif n > 5 \then n - 5 \myelse \bot(\bot(n+13))\\
                  &= \myif n > 5 \then n - 5 \myelse \bot\\
                  &= \myif n > 5 \then n - 5 \myelse \bot(n+8)\\
                  &= G\ \bot\ n
    \end{align*}
    \item Case $f,g$ with $f \neq \bot \neq g$:
    \begin{align*}
        F\ f\ n &= \myif n > 5 \then n - 5 \myelse f(f(n+13))\\
              &= \myif n > 5 \then n - 5 \myelse f(\myif n + 13 > 5 \then n + 13 - 5 \myelse f(f(n+13+13)))\\
              &= \myif n > 5 \then n - 5 \myelse f(n+13-5)\\
              &= \myif n > 5 \then n - 5 \myelse f(n+8)\\
              &= \myif n > 5 \then n - 5 \myelse g(n+8) \text{ by I.H.}\\
              &= G\ g\ n
    \end{align*}
\end{itemize}

\section{Exercise 10.3.14}
Let $D$ be the set of natural numbers. Prove that the fixed-point operator
\begin{align*}
    fix &: (D \rightarrow D) \rightarrow D \text{ where }\\
    fix &= \lambda F.\lub\{F^i(\bot)|i\geq 1\}
\end{align*}
is monotonic and continuous.

\paragraph{Solution} We only show continuity, since it implies monotonicity.
That is, wee need to show, that for any ascending chain $f_0 \subseteq f_1 \subseteq f_2 \subseteq \cdots$
we have $\fix(\lub\{f_i\ |\ i \geq 0\}) = \lub\{\fix(f_i)\ |\ i \geq 0\}$.
\begin{align*}
    &\mathbin{\hphantom{=}} \fix(\lub\{f_i\ |\ i \geq 0\})\\
    &=\lub\{(\lub\{f_i\ |\ i \geq 0\})^j\ |\ j \geq 0\}     &\text{ by def. of } \fix\\
    &=\lub\{\lub\{f_i^j\ |\ i \geq 0\}\ |\ j \geq 0\}       &\forall i. f_i \text{ continuous}\\
    &=\lub\{\lub\{f_i^j\ |\ j \geq 0\}\ |\ i \geq 0\}       &\lub \text{ associative}\\
    &=\lub\{\fix(f_i)\ |\ i \geq 0\}                        &\text{ by def. of } \fix
\end{align*}

\section{Exercise 10.3.16} Contextfree
grammars can be viewed as systems of equations where the
nonterminals are regarded as variables (or unknowns) over sets of strings;
the solution for the start symbol yields the language to be defined. In
general, such an equation system has solutions that are tuples of sets,
one for each nonterminal. Such solutions can be regarded as fixed points
in that when they are substituted in the right-hand side, the result is
precisely the solution again. For example (using ε for the null string), the
grammar
\begin{align*}
    &A \Coloneqq aAc\ |\ B\\
    &B \Coloneqq bB\ |\ C\\
    &C \Coloneqq \epsilon\ |\ C
\end{align*}
corresponds to the transformation on triples $\langle X,Y,Z \rangle$ of sets defined
by
\begin{equation*}
    f(\langle X,Y,Z \rangle) = \langle \{a\} \bullet X \bullet \{c\} \cup Y, \{b\} \bullet Y \cup Z, \{\epsilon\} \cup Z \rangle \text{,}
\end{equation*}
whose fixed point $\langle A,B,C \rangle$ then satisfies the set equations
\begin{align*}
    &A \Coloneqq \{a\} \bullet A \bullet \{c\}\ |\ B\\
    &B \Coloneqq \{b\} \bullet B\ |\ C\\
    &C \Coloneqq \epsilon\ |\ C
\end{align*}
for appopriate $A,B,C \subseteq \{a,b,c\}^*$. For instance, the quations above are satisfied by the sets
$A = \{a^nb^*c^n\ |\ n \geq 0\}, B = b^*, C = \{\epsilon\}$.
Show that the equation system corresponding to the grammar above
has more than one possible solution so that simply seeking an arbitrary
solution is insufficient for formal language purposes. However, the \textit{least}
fixed point solution provides exactly the language normally defined by
the grammar. Illustrate how the first few steps of the ascending chain in
the fixed-point construction lead to the desired language elements for
the grammar above, and discuss the connection with derivations in the
grammar.\\
\textit{Note}: We have the natural partial order for tuples of sets where
\begin{equation*}
    \langle S_1, \cdots, S_k \rangle \subseteq \langle T_1, \cdots, T_k \rangle \myif S_i \subseteq T_i \text{ for all } i, 1 \leq i \leq k
\end{equation*}

\paragraph{Solution}

We show the first part by providing just another fixed point of $f$.
Let $\Sigma = \{a,b,c\}$ and $P = \langle \Sigma^*, \Sigma^*, \Sigma^* \rangle$.
Then $P$ is a fixed point of $f$:\\
$f(\langle \Sigma^*, \Sigma^*, \Sigma^* \rangle) = \langle
  \{a\} \bullet \Sigma^* \bullet \{c\} \cup \Sigma^*,
  \{b\} \bullet \Sigma^* \cup \Sigma^*,
  \{\epsilon\} \cup \Sigma^*
\rangle = \langle \Sigma^*, \Sigma^*, \Sigma^* \rangle$

Now, lets observe the first few steps of the fixed-point construction.

\newcommand{\fa}[2]{\ensuremath{\{a\} \bullet #1 \bullet \{c\} \cup #2}}
\newcommand{\fb}[2]{\ensuremath{\{b\} \bullet #1 \cup #2}}
\newcommand{\fc}[1]{\ensuremath{\{\epsilon\} \cup #1}}

\begin{figure}[H]
    \begin{tabular}{l | c | c | c }
        & X      & Y      & Z\\
        \hline
        $f^0$ & $\bot$ & $\bot$ & $\bot$\\
        $f^1$ & $\bot$ & $\bot$ & $\fc{\bot}$\\
        $f^2$ & $\bot$ & $\fc{\bot}$ & $\fc{\bot}$\\
        $f^3$ & $\fc{\bot}$ & $\{\epsilon,b\} \cup \bot $ & $\fc{\bot}$\\
        $f^4$ & $\{\epsilon,b,ac\} \cup \bot$ & $\{\epsilon,bb\} \cup \bot $ & $\fc{\bot}$\\
        $f^5$ & $\{\epsilon,bb,abc,aacc\} \cup \bot$ & $\{\epsilon,bbb\} \cup \bot $ & $\fc{\bot}$\\
        $f^6$ & $\{\epsilon,bbb,abbc,aabcc,aaaccc\} \cup \bot$ & $\{\epsilon,bbbb\} \cup \bot $ & $\fc{\bot}$\\
    \end{tabular}    
\end{figure}

In general, we observe that $f^i(\langle \bot,\bot,\bot \rangle) = \langle \{ w\ |\ A \Rightarrow^i w\} \cup \bot,\{ w\ |\ B \Rightarrow^i w\} \cup \bot, \{ w\ |\ C \Rightarrow^i w\} \cup \bot \rangle$. Where $T \Rightarrow^i w$ means, that $w$ can be derived with exactly $i$ steps from the nonterminal $T$.

\end{document}
